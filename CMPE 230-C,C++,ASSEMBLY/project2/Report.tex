\documentclass[11pt]{article}
\usepackage{titlesec}
\usepackage{fancyhdr}
\usepackage{graphicx}
\usepackage{listings}
\usepackage{color}
\usepackage{amsmath}
\usepackage{hyperref}
\usepackage{enumitem}
\usepackage[a4paper, margin=1in]{geometry}

\titleformat{\section}{\normalfont\Large\bfseries}{\thesection}{1em}{}
\titleformat{\subsection}{\normalfont\large\bfseries}{\thesubsection}{1em}{}

\pagestyle{fancy}
\fancyhf{}
\rhead{Spring 2025}
\lhead{Systems Programming}
\cfoot{\thepage}

\title{Project Report \\ EvoChirp: Birdsong Evolution Simulator \\ \large{Systems Programming -- Spring 2025}}
\author{Ali Ayhan Gunder, 2021400219 \\ Yunus Emre Ozturk, 2022400204}
\date{May 11, 2025}

\begin{document}

\maketitle

\tableofcontents

\section{Introduction}
This project implements a GNU Assembly-based simulator called \textbf{EvoChirp}, which models the evolution of bird songs across generations. Inspired by natural avian communication systems, the program interprets a structured input expression composed of species, notes, and operators to simulate species-specific song transformations. Each bird species---Sparrow, Warbler, and Nightingale---responds to the same input in uniquely defined evolutionary ways. This program emphasizes control flow, pointer arithmetic, string processing, and low-level system programming.

\subsection{Problem Statement}
Birdsong plays a vital role in mating, signaling, and communication. In this simulation, the input represents a bird species and an evolutionary song expression. The program applies each operator to the current song based on the species rules and outputs the song's state after every transformation.

\subsection{Objectives}
\begin{itemize}[noitemsep]
    \item Develop a GNU assembly program to simulate song evolution.
    \item Implement per-species behavior for four operators: + (merge), - (reduce), * (repeat), H (harmonize).
    \item Print the song state after each generation of transformation.
    \item Ensure correctness with input parsing, string manipulation, and memory-safe operations.
\end{itemize}

\subsection{Motivation}
The EvoChirp project challenges students to handle real-world string processing in assembly, using architecture-specific conventions while simulating natural phenomena. It encourages clean, modular design even in a low-level language and reinforces understanding of memory management and control logic.

\section{Problem Description}
The input format is a single line string structured as:
\begin{quote}
\texttt{<Species> <Note> <Note> ... <Operator> <Operator> ...}
\end{quote}

For each valid line, the program must:
\begin{itemize}[noitemsep]
    \item Identify the species (Sparrow, Warbler, or Nightingale).
    \item Parse the song expression (notes + operators).
    \item Apply each operator to the song according to species-specific rules.
    \item Output the resulting song after each operation.
\end{itemize}

The simulator enforces a strict grammar and assumes no malformed input.

\section{Methodology}
\subsection{Data Structures}
The notes array is implemented as a statically allocated block in the stack frame, with a custom format that includes an index counter and 128-entry note buffer. Each entry is 8 bytes (to support string operations).

\subsection{Key Functions}
\begin{itemize}[noitemsep]
    \item \texttt{get\_species}: Maps species name to species code (0--2).
    \item \texttt{op\_plus}, \texttt{op\_minus}, \texttt{op\_star}, \texttt{op\_h}: Species-specific implementations for each operator.
    \item \texttt{softness}: Determines note softness for Sparrow's reduction rule.
    \item \texttt{process\_input\_line}: Tokenizes input, dispatches operations, prints each generation.
\end{itemize}

\subsection{Operator Semantics}
The operator meanings vary by species. For instance:
\begin{itemize}[noitemsep]
    \item \textbf{Sparrow}: Merges with hyphen, transforms notes globally on H.
    \item \textbf{Warbler}: Repeats last notes or appends trill.
    \item \textbf{Nightingale}: Expands entire song or mirrors last three notes.
\end{itemize}

\section{Implementation}
\subsection{Code Structure}
The code is modular and consists of:
\begin{itemize}[noitemsep]
    \item Global constants (.rodata) for strings and formats.
    \item Dedicated functions for each operator.
    \item Main loop that reads input and handles exit conditions.
    \item Tokenized processing with register-safe logic and pointer manipulation.
\end{itemize}

\subsection{Sample Code Snippet}
\begin{lstlisting}[language={[x86masm]Assembler}, basicstyle=\ttfamily\footnotesize]
# Constants and format strings
.section .rodata
bird_species_sparrow:       .string "Sparrow"               
# Name of Sparrow species
bird_species_warbler:       .string "Warbler"               
# Name of Warbler species
bird_species_nightingale:   .string "Nightingale"           
# Name of Nightingale species
format_string_combine:      .string "%s-%s"                 
# Format for combining notes
note_c:                     .string "C"                     
# Chirp note
note_t:                     .string "T"                     
# Trill note
note_d:                     .string "D"                     
# Deep call note
space_char:                 .string " "                     
# Space character for tokenizing
error_unknown_species:      .string "Unknown species: %s\n" 
# Error message for invalid species
format_generation:          .string "%s Gen %d:"            
# Format for generation output
format_note:                .string " %s"                   
# Format for individual notes
newline:                    .string "\n"                    
# New line character
cmd_exit:                   .string "exit"                  
# Exit command
cmd_quit:                   .string "quit"                  
# Quit command

# Function: op_plus - Merge notes according to species rules
.globl  op_plus
op_plus:
        pushq   %rbp                  # Set stack frame
        movq    %rsp, %rbp
        subq    $48, %rsp             # Allocate space for local variables
        movq    %rdi, -40(%rbp)       # Store pointer to notes array
        movl    %esi, -44(%rbp)       # Store species code
        
        # Check if there are at least two notes
        movq    -40(%rbp), %rax
        movl    1024(%rax), %eax      # Load current note count
        cmpl    $1, %eax              # If notes <= 1, skip operation
        jle     .op_plus_insufficient_notes
        
        # Get pointer to last note
        movq    -40(%rbp), %rax
        movl    1024(%rax), %eax
        subl    $1, %eax
        cltq
        leaq    0(,%rax,8), %rdx
        movq    -40(%rbp), %rax
        addq    %rdx, %rax
        movq    %rax, -8(%rbp)        # Pointer to last note
        
        # Get pointer to second last note
        movq    -40(%rbp), %rax
        movl    1024(%rax), %eax
        subl    $2, %eax
        cltq
        leaq    0(,%rax,8), %rdx
        movq    -40(%rbp), %rax
        addq    %rdx, %rax
        movq    %rax, -16(%rbp)       # Pointer to second last note
        # ... (continues with merge logic)

# Function: op_star - Repeat notes based on species rules
.globl  op_star
op_star:
        pushq   %rbp                  # Set stack frame
        movq    %rsp, %rbp
        subq    $32, %rsp             # Allocate space for local variables
        movq    %rdi, -24(%rbp)       # Store pointer to notes array
        movl    %esi, -28(%rbp)       # Store species code
        
        # Check if notes array is empty
        movq    -24(%rbp), %rax
        movl    1024(%rax), %eax      # Load note count
        testl   %eax, %eax            # Check if zero notes
        je      .op_star_empty_array  # Skip if no notes
        
        # Select operation based on species
        cmpl    $2, -28(%rbp)         # Nightingale species?
        je      .op_star_nightingale
        cmpl    $2, -28(%rbp)
        ja      .op_star_default
        cmpl    $0, -28(%rbp)         # Sparrow species?
        je      .op_star_sparrow
        cmpl    $1, -28(%rbp)         # Warbler species?
        je      .op_star_warbler
        jmp     .op_star_default
        # ... (continues with repeat logic)

# Function: process_input_line - Process and tokenize input line
.globl  process_input_line
process_input_line:
        pushq   %rbp                      # Set stack frame
        movq    %rsp, %rbp
        subq    $1872, %rsp               # Allocate buffer space
        movq    %rdi, -1864(%rbp)         # Store input line pointer

        # Copy input safely to buffer
        movq    -1864(%rbp), %rcx
        leaq    -304(%rbp), %rax
        movl    $256, %edx                # Max length 256 chars
        movq    %rcx, %rsi
        movq    %rax, %rdi
        call    strncpy
        movb    $0, -49(%rbp)             # Add null terminator
        
        # Split input line into tokens
        movl    $0, -4(%rbp)              # Token counter start at 0
        leaq    -304(%rbp), %rax
        movl    $space_char, %esi         # Space char as separator
        movq    %rax, %rdi
        call    strtok
        movq    %rax, -16(%rbp)           # First token pointer
        jmp     .process_tokens
        # ... (continues with token processing logic)

# Function: main - Entry point and main loop
.globl  main
main:
        pushq   %rbp                      # Set stack frame
        movq    %rsp, %rbp
        subq    $256, %rsp                # Buffer for input line

command_loop:
        # Read input line from user
        movq    stdin(%rip), %rdx         # stdin stream
        leaq    -256(%rbp), %rax          # Input buffer address
        movl    $256, %esi                # Max chars to read
        movq    %rax, %rdi                # Input buffer
        call    fgets                     # Read user input line
        testq   %rax, %rax                # Check EOF or error
        je      finish_program            # Exit if EOF or error detected
        # ... (continues with checking for exit and calling processing functions)

\end{lstlisting}

\section{Results}
The program was tested using all five official test cases provided in the project description. After addressing minor bugs in memory indexing and species identification, the program passed \textbf{100\% of the grader tests}. It produces the correct output format and transformations for all operators across all species.

\section{Discussion}
The primary challenge was emulating high-level string logic using assembly language. Managing memory manually for variable-length strings, ensuring register correctness, and validating grammar compliance added significant complexity. The project highlights the tradeoff between control and convenience in low-level languages.

Future improvements could include:
\begin{itemize}[noitemsep]
    \item Dynamic memory allocation for note arrays.
    \item Additional operator support or species customization.
    \item Full input grammar checking and recovery.
\end{itemize}

\section{Conclusion}
EvoChirp successfully implements a text-based bird song evolution simulator in assembly. It reinforces foundational systems programming concepts including memory management, stack-based computation, and control flow. The program adheres strictly to specification and output format, and demonstrates a deep understanding of x86\_64 GNU Assembly practices.

\section*{References}
\begin{itemize}[noitemsep]
    \item CMPE230 Assignment 2 PDF
    \item Piazza Q/A Discussions
    \item Lecture Slides and Notes
    \item GNU x86\_64 Assembly Documentation
\end{itemize}

\section*{AI Assistants}
AI tools such as ChatGPT were used for the following:
\begin{itemize}[noitemsep]
    \item Clarifying assembly syntax and debugging segmentation faults
    \item Understanding x86\_64 calling conventions
    \item Explaining GNU toolchain behavior and register usage
    \item Formatting and grammar-checking LaTeX documentation
\end{itemize}

All implementation decisions and coding were performed by the project authors.

\end{document}
